\documentclass{scrartcl}
\usepackage[
headfootpackage=scrpage,
inputlineshead=outside,
inputlinesfoot=outside,
dilinputfiles=true,
inputfilesrule=true,
inputfloatname=true,
]{draftinputlines}
\usepackage{listings}
\lstloadlanguages{[LaTeX]TeX} 
\lstset{% 
    basicstyle=\ttfamily, 
    commentstyle=\itshape\ttfamily\small, 
    showspaces=false, 
    showstringspaces=false, 
    breaklines=true, 
    breakautoindent=true, 
    captionpos=t,
    frame=single,
    escapeinside={(*}{*)},
} 
\lstMakeShortInline{|}
\usepackage{enumitem}
\setdescription{font=\ttfamily,leftmargin=4em}
% \xdef\thefilename{\currfilename}
% \chead{\texttt{\thefilename}\xdef\thefilename{\currfilename}}
% \cfoot{\texttt{\currfilename}}
\begin{document}
\section{Introduction}
This package adds the input lines of various parts of your LaTeX
document as marginpars.
It can also add the first and last input lines read to the header and
footer respectively.

The marginpars float, and I can't guarantee they'll be entirely
accurate if lots are bunched together.\marginnote{Rightie}
Also, the begin and end page numbers might not be totally accurate.
But they're still useful (I hope).

The numbers that float with tables and the like do not float, so they
will overlap other numbers in some circumstances.
This should be rare since floats should have their own space.

\section{Examples}
This demonstrates some of the capacities of the package.

\begin{itemize}
\item One
\item Two
\item Three
\end{itemize}
\vspace{5cm}
\begin{enumerate}
\item One
\item Two
\item Three
\end{enumerate}

\begin{table}
  \centering
  \begin{tabular}{lr}
    This & Floats \\
    So & Does \\
    Margin & Note
  \end{tabular}
  \caption{This is a float from the input file}
\end{table}



\section{Options}
Here are the options that are currently defined and a description of
what they do.

\paragraph{\lstinline+headfootpackage+}
This sets what package is used to do the headers and footers.
Set it to whatever is setting the headers and footers in your
document.
The options are:
\begin{description}[labelindent=1cm]
\item[scrpage] Uses KOMA script's |scrpage2|
\item[fancyhdr] Uses |fancyhdr|
\item[memoir] Uses |memoir|. Awkwardly, given the way |memoir|'s
  header and footer mechanisms work, this will overwrite any current
  headers you have set.
  So this just sets a page number somewhere where it isn't setting an
  input line number.
  If you want to add the input line number to your existing headers
  instead, you can do that with these two macros: 
  \textsc{These macros don't exist yet. 
    When I make memoir work, I will fix this.}
  \item[plain] Use the plain TeX macros.
\end{description}

\paragraph{\lstinline+inputlineshead,inputlinesfoot+}

The print the line number at the start of the page at the top and the
end of the page at the bottom.
The options available are the same for both, though they can be set separately. 
They are:
\begin{description}[labelindent=1cm]
\item[inside] Print number on the inside (same as |left| for onesided
  documents)
\item[outside] Print number on the outside (same as |right| for
  onesided documents)
\item[center] Print number on in the center
\item[left] Print number on the left
\item[right] Print number on the right
\item[off] Don't print the numbers
\end{description}

They aren't necessarily 100\% accurate given the asynchronous way that
the page builder works, but they are probably good enough for most purposes.

\paragraph{\lstinline+dilfont+}

This defines what font the input line numbers will appear in.
The default is |\ttfamily|.

\paragraph{\lstinline+environmentline+}

This option takes a list of environment names.
Those environments listed will then print the line number where they
start.

\paragraph{Switches}

The following is a list of Boolean (true/false) options and what they
control.
\begin{description}
\item[sectionline] Determines whether each section will print the line
  it is printed on.
\item[captionline] Determines whether captions to figures, tables and
  so on, will print their input line number.
  I expect this will break with subcaptions.
  These marginpars don't float.
\item[inputfloatname] Prints the name of the file the float is from.
  If |captionline| is false, this has no effect.
\item[dilbasicenvlines] This basically passes a bunch of basic
  environments to |environmentline|, namely:
  |itemize,enumerate,description,quote,displaymath,equation,equation*,quotation,verse,abstract,flushleft,center,flushright,eqnarry,minipage|
\item[dilamsmathenvlines] Like |dilbasicenvlines| this makes some
  |amsmath| environments print their input line numbers.
  For now it only actually does this for |align| and |align*|.
  Most of the other environments are supposed to be use \emph{inside}
  another mathmode environment, and so the outer mathmode
  environment's start should give enough of an indication of where
  this environment is.
\item[dilinputfiles] Prints the name and input line number of files
  included with |\input| and |\include|.
\item[dilinputfilesrule] Prints rules above the start and below the
  end of an included file, so it's obvious where in the page it starts
  and ends.
  If |dilinputfiles| is false, this doesn't do anything.
\end{description}




\end{document}

%%% Local Variables: 
%%% mode: latex
%%% TeX-master: t
%%% End: 
